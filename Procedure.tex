\documentclass{article}
\usepackage{graphicx}
\usepackage{amsmath}
\usepackage{hyperref}

\title{Simulating Heat Equation in a Rod and Plate}

\begin{document}

\maketitle

\section{Introduction}
The heat equation is a partial differential equation that describes how the distribution of temperature in a given region changes over time. In this report, we will discuss the procedure for simulating the heat equation in both a rod and a plate using numerical methods.

\section{Heat Equation}
The heat equation for a one-dimensional rod and a two-dimensional plate can be expressed as follows:

\subsection{One-Dimensional Heat Equation (Rod)}
\begin{equation}
    \frac{\partial u}{\partial t} = \alpha \frac{\partial^2 u}{\partial x^2}
\end{equation}

\subsection{Two-Dimensional Heat Equation (Plate)}
\begin{equation}
    \frac{\partial u}{\partial t} = \alpha \left( \frac{\partial^2 u}{\partial x^2} + \frac{\partial^2 u}{\partial y^2} \right)
\end{equation}

where $u(x, y, t)$ represents the temperature distribution at position $(x, y)$ and time $t$, and $\alpha$ is the thermal diffusivity.

\section{Numerical Solution}
We will use finite difference methods to approximate the solution to the heat equation numerically. The basic idea is to discretize the spatial domain and approximate the derivatives using the finite difference approximations.

\subsection{One-Dimensional Finite Difference Method}
For the one-dimensional case, we can use the explicit forward difference method to discretize the heat equation as follows:

\begin{equation}
    u_{i}^{n+1} = u_{i}^{n} + \frac{\alpha \Delta t}{\Delta x^2} \left( u_{i+1}^{n} - 2u_{i}^{n} + u_{i-1}^{n} \right)
\end{equation}

where $u_{i}^{n}$ represents the temperature at position $x_i$ and time $t_n$, and $\Delta x$ and $\Delta t$ are the spatial and temporal discretization steps, respectively.

\subsection{Two-Dimensional Finite Difference Method}
For the two-dimensional case, we can use similar finite difference approximations for both spatial dimensions:

\begin{multline}
    u_{i,j}^{n+1} = u_{i,j}^{n} + \frac{\alpha \Delta t}{\Delta x^2} \left( u_{i+1,j}^{n} - 2u_{i,j}^{n} + u_{i-1,j}^{n} \right) \\
    + \frac{\alpha \Delta t}{\Delta y^2} \left( u_{i,j+1}^{n} - 2u_{i,j}^{n} + u_{i,j-1}^{n} \right)
\end{multline}

\section{Conclusion}
In this report, we have outlined the procedure for simulating the heat equation in both a rod and a plate using numerical methods. These simulations can be implemented in various programming languages such as Python or MATLAB and used to study heat transfer phenomena in different physical systems.

\end{document}

